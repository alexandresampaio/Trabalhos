\begin{center}
\textbf{RESUMO}
\end{center}
\singlespacing

\noindent Este trabalho descreve sobre uma aplicação desenvolvida no Instituto Federal Fluminense com o intuito de mostrar uma base de dados NoSQL em comunicação com um Framework ágil de desenvolvimento web para aramazenar dados simultaneos
em rede. A ferramenta apresenta uma forma de upload rapida e fácil de arquivos e manipulação dos mesmos, por exemplo.\\

\noindent Ainda é descrito no trabalho um pouco sobre as tecnologias utilizadas e/sobre os processos de desenvolvimento da aplicação, e outras ferramentas que foram utilizadas a fim de tornar possível a existência das futuras funcionalidades. Através da 
dedicação no Núcleo de Pesquisa e Engenharia de Software(NES), foi possível implementar as funcionalidades de manipulação de formatos correspondentes imagens e PDF; Encontram-se ainda, neste trabalho, a estrutura da ferramenta, conceitos básicos empregados para formação da ferramenta e sua integração com a arquitetura linux, bem como conceitos da linguagem Ruby, a qual foi utilizada para seu desenvolvimento
cujo Web Framework é o Rails e base de dados NoSQL é o MongoDB. \\

\noindent PALAVRAS-CHAVE:  Serviço Web, Software livre, NoSQL, Meta-Programming, Ruby on Rails
